\documentclass[11pt]{article}
\usepackage{color, array, graphics}
\usepackage{amsmath}
\usepackage{amssymb}
\usepackage{enumerate}
\usepackage{mathtools}
\usepackage{fullpage}
\usepackage{graphicx}
\usepackage{float}
\usepackage{listings}
\usepackage[utf8]{inputenc}

\begin{document}
\lstset{stringstyle=\ttfamily,
	showstringspaces=false,
	basicstyle=\small}

\begin{center} Alexander Garcia \hfill June 28, 2017 \\ Assignment-5 \end{center}

\medskip

\begin{enumerate}

	\item $$I = \int_{-1}^{1}e^{-2x} dx$$

	\begin{enumerate}

		\item Approximation using the composite trapezoidal rule.

		The general formula for the composite trapezoidal rule is
		\[
		I(f) = \frac{h}{2}[f(a)+f(h)+2\sum_{j=1}^{k-1}f(a+jp)]
		\]
		where $h$ is the width of the total interval, $p$ is the node width, and $k$ is the total number of nodes. In this case, $h=p = 0.5$, $k=4$, and $f = e^{-2x}$.

		Then, according to the rule,

		\[
		I(f) = \frac{0.5}{2}[f(-1) + f(1) + 2\sum_{j=1}^{3}f(a+jp)]
		\]

		or

		\[
		I(f) = \frac{1}{4}[e^{-1} + e + 2[e^{-1+0.5} + e^{-1+1} + e^{-1+1.5}]]
		\]

		When the expression is evaluated, we get the composite trapezoidal approximation of the integral to be
		\[
		I(f) \approx 2.399166\ldots
		\] \

		We know that the error for this formula can be expressed by $-\frac{h^2}{12}(b-a)f''(\eta)$, where $\eta$ is an unknown location within the interval. Rather than use an unknown location, we can bound the error by the maximum of the second derivative over the interval.

		\[f''(x) = 4e^{-2x} \leq 4e^2\ \ x\in[-1,1]\]

		Therefore, the maximum error of this approximation is
		\[
		\frac{0.5^2}{12}(1-(-1))(4e^2) \approx 1.231509\ldots
		\]

		which makes our result not very significant. \\

		\item Approximation using composite Simpson's rule

		We can define composite Simpson's rule in this case as the following
		\[
		I(f) = \frac{h}{3}\sum_{j=2}^{8}(f(a+(j-2)h) + 4f(a+(j-1)h) + f(a+jh))
		\]

		In our case, $k = 4$, and $p = \frac{b-a}{k}$, so $h = \frac{p}{2}$. We can write out the full summation then as \\

		$\frac{h}{3}$[
		\begin{tabular}{ll}
			$f(-1) + 4f(-0.75) + f(-0.5)$ & $+$ \\
			$f(-0.5) + 4f(-0.25) + f(0)$ & $+$ \\
			$f(0) + 4f(0.25) + f(0.5)$ & $+$ \\
			$f(0.5) + 4f(0.75) + f(1)$
		\end{tabular}
		] \\

		The overall error of Simpson's rule approximation can be written as

		\[
		E \leq -\frac{b-a}{180}h^4max(f''(x))\ \ x\in[-1,1]
		\]

		We can easily see that the $max(f''(x)) = 16e^2$ for our interval, so the upper bound on the error is

		We can then calculate the upper bound on the error.

		\[
		E \leq -\frac{2}{180}(0.25^4)(16e^2)
		\]
		\[
		E \leq -0.00513129\ldots
		\]

		This makes composite Simpson's approximation a much better estimate of the integral than composite trapezoidal approximation. \\

		\item For the composite trapezoidal rule, we are given that the error is defined by

		\[
		|E| \leq \frac{h^2}{12} (b-a) max(f''(x)) < 10^{-6}\ \ x\in[a,b]
		\]

		As we are simply solving for the node spacing $h$, we can rearrange this equation to gain an expression for its value.

		\[
		h  < \sqrt{\frac{12\delta}{(b-a)M}}
		\]

		where

		$M=max(f''(x))\ x\in[-1,1] = 4e^2$

		$\delta=10^{-6}$, and

		$b-a=2$.

		When these values quantities are used, we get

		\[
		h < (\sqrt{\frac{12*10^{-6}}{2*4e^2}} \approx 0.000450558\ldots)
		\] \

		\item We have a similar expression for the error of Simpson's rule

		\[
		|E| \leq \frac{b-a}{180}h^4max(f^{(iv)}(x)) < 10^{-6}\ \ x\in[a,b]
		\]

		We can again rearrange this equation to solve for $h$, the minimum node spacing to ensure an error of $\leq 10^{-6}$

		\[
		h < (\frac{180\delta}{(b-a)M})^{1/4}
		\]

		where $M=max(f^{(iv)}(x))\ \ x\in[-1,1] = 16e^2$

		$\delta = 10^{-6}$, and

		$b-a=2$.

		This inequality gives us a minimum node spacing of

		\[
		h < ((\frac{180*10^{-6}}{2*16e^2})^{1/4} \approx 0.0295381\ldots)
		\] \

	\end{enumerate}

	\item Holmes 6.18

	\begin{enumerate}[(a)]

	\item $I_S(n) = \frac{2}{3}I_T(n) + \frac{1}{3}I_M(n/2)$

	The approximation of this integral must take place over an interval $[a,b]$, the size of which determines the node spacing $h$.
	For $n$ nodes, $h=\frac{b-a}{n}$, and for $n/2$ nodes, $h = 2\frac{b-a}{n}$. Going forward, we use these values of $h$ for $I_T(n)$ and $I_M(n)$ respectively. \\

	We know the definition of $I_M(n)$ as
	\[
	I_M(n) = h[f(a+h/2) + f(a+h) + f(a+3h/2) + f(a+2h) + \ldots + f(a+(n-1)h/2)]
	\] \

	Considering the fact that $h=2\frac{b-a}{n}$ when we take $I_M(n/2)$, we can rewrite this equation as
	\[
	I_M(n/2) = 2\frac{b-a}{n}[f(a+2\frac{b-a}{n}) + f(a+4\frac{b-a}{n}) + \ldots]
	\]
	and
	\[
	\frac{1}{3}I_M(n/2) = \frac{2(b-a)}{3n}[f(a+2\frac{b-a}{n}) + \ldots]
	\] \

	We also know the definition of $I_T(n)$
	\[
	I_T(n) = \frac{b-a}{2n}[f(a) + f(b) + 2f(a+\frac{b-a}{n}) + 2f(a+2\frac{b-a}{n}) + 2f(a+3\frac{b-a}{n})+\ldots]
	\]
	and
	\[
	\frac{2}{3}I_T(n) = \frac{b-a}{3n}[f(a) + f(b) + 2f(a+\frac{b-a}{n}) + 2f(a+2\frac{b-a}{n}) + \ldots]
	\] \

	These two sequences share terms that are of the form $f(a+2i\frac{b-a}{n})$, meaning the even terms are seen twice. In $I_T(n)$, there are also multiplied by a factor of 2, meaning that in total, each even term appears 4 times. The odd terms in $I_T(n)$, then, are only found in $I_T(n)$, but are still multiplied by 2, so each odd term appears 2 times. Each end point $f(a), f(b)$ only appears in $I_T(n)$, so each has a weight of 1. When the two sequences are combined, we get
	\[
	\frac{(b-a)}{3n}[f(a) + f(b) + 4f(x_{even}) + 2f(x_{odd})]
	\]
	where $f(x_{odd}) = f(a+odd\frac{b-a}{n})$ and $f(x_{even}) = a+even\frac{b-a}{n}$. This is the definition of composite Simpson's rule. \\

	\item $I_S(n) = \frac{4}{3}I_T(n) - \frac{1}{3}I_M(n/2)$

	A very similar idea is used to demonstrate this alternate definition of composite Simpson's rule.

	\[
	\frac{1}{3}I_M(n/2) = \frac{2(b-a)}{3n}[f(a+2\frac{b-a}{n}) + \ldots]
	\]

	and

	\[
	\frac{4}{3}I_T(n) = \frac{2(b-a)}{3n}[f(a) + f(b) + 2f(a+\frac{b-a}{n}) + 2f(a+2\frac{b-a}{n}) + \ldots]
	\]

	Again, once the difference is taken, the definition of composite Simpson's rule appears.

	\[
	\frac{4}{3}I_T(n) - \frac{1}{3}I_M(n/2) = \frac{b-a}{3n}[f(a) + f(b) + 4f(x_{even}) + 2f(x_{odd})]
	\] \

	\end{enumerate}

	\item $Q_3(f) = \frac{4h}{3}[2f(h)-f(2h)+2f(3h)]\ \ 0\leq x \leq 4h$

	\begin{enumerate}[(a)]

	\item Given that this approximation uses 3 nodes, and therefore uses three function evaluations to approximate the integral, we can say that $Q_3(f)$ is of precision $3$. This means $Q_3(f)$ can approximate integrals exactly for polynomials up to and including degree 3. \\

	\item We know that in general, the formula for the error in an odd degree Quadrature rule is
	$$E_n =  Kh^{n+1}p^n(\eta)$$
	where $h$ is the node spacing, $p^n$ is a polynomial of degree $n+1$, and $\eta$ is an unknown location on the interval. In order to find the value of $K$, we chose the function $f(x) = x^4$, where $f^{(iv)}(x) = 24$. It is known that $$\int x^4 = \frac{x^5}{5}$$ so $$\int_{0}^{4h}x^4 = \frac{(4h)^5}{5}$$

	We also know the approximation, as it was given to be
	\[
	\frac{4h}{3}[2f(h)-f(2h)+2f(3h)]
	\]

	If we expand each side as a Taylor series, we get
	\[
	F(x) = \int f(x) = F(a) + hF'(a) + \frac{h^2}{2}F''(a) + \frac{h^3}{3!}F'''(a) + \ldots
	\]
	where $a$ is given as 0. We know that $F(0)$ is zero, since $F$ is a one term polynomial. $F$ can now be rewritten in terms of $f$, as the previously unknown term $F(a)$ is now gone.

	\[
	F(x) = (x-a)f(a) + \frac{(x-a)^2}{2!}f''(a) + \frac{(x-a)^3}{3!}f'''(a) + \frac{(x-a)^4}{4!}f^{(iv)}(a)
	\]

	What we are looking for is $F(4h)$. We use the Taylor expansion of $F$ to find this value. As we know that $x-a = 4h$, we can make this substitution as well.

	\[
	F(4h) = 4hf(a) + \frac{(4h)^2}{2!}f'(a) + \frac{(4h)^3}{3!}f''(a) + \frac{(4h)^4}{4!}f^{(iv)}(a)
	\]

	Remembering that $F(a)$ was being approximated by $\frac{4h}{3}[2f(h)-f(2h)+2f(3h)]$, we can expand each $f$ term in the approximation to achieve

	\[
	F^*(h) = \frac{4h}{3}[2f(a) +
	\]
	\[
	[f(a) + 2hf'(a) + \frac{4h^2}{2!}f''(a) + \frac{8h^3}{3!}f'''(a)+\ldots] -
	\]
	\[
	2[f(a) + 3hf'(a) + \frac{9h^2}{2!}f''(a) + \frac{27h^3}{3!}f'''(a) + \ldots]] + E
	\]

	If we simply compare the $f^{(iv)}(x)$ terms in each of the Taylor expansions, we see that the term for $F$ is
	$\frac{32h^5}{5!}f^{(iv)}(a)$
	, and the term for $F^*$ is $\frac{4h}{3}[\frac{(2h)^5}{5!} - \frac{(3h)^5}{5!}]f^{(iv)}(a) = -\frac{211h^5}{5!}f^{(iv)}(a)+\ldots+E$

	When we take the difference between the actual and the approximate terms, we finally see E

	\[
	E = [\frac{32h^5}{5!}+\frac{211h^5}{5!}]f^{(iv)}(a) = \frac{241h^5}{5!}+\ldots
	\]

	or, if we want to truncate the sequence by using an unknown $a$,

	\[
	E = \frac{241h^5}{5!}f^{(iv)}(\eta)
	\]

	\end{enumerate}

	\item

\end{enumerate}

\end{document}
