% \input{../def}
\input{../def}

\begin{document}

\bc Alexander Garcia \hfill May 30, 2017 \\ Assignment-1\ec

\bigskip

\benum

% Problem 1 ----------------------------------------------------------------------------------------------------
\item Let $f(x) = e^{x}\cos {2x}.$
\benum
\item

	The definition of the Taylor polynomial is as follows:

	$$T_n(x) = \sum_{k=0}^{n}{\frac{(x-x_0)^k}{k!}f^{(k)}(x_0)}$$

	In this case, $n=4$, and $x_0 = 0$.

	We start by finding $f^{(1)}(x), f^{(2)}(x), f^{(3)}(x)$, and $f^{(4)}(x)$ \\

	$f^{(1)}(x) = -2e^xsin(2x)+e^xcos(2x) = e^x(cos(2x)-2sin(2x))$

	$f^{(2)}(x) = e^x(-2sin(2x)-4cos(2x))+e^x(cos(2x)-2sin(2x)) = -e^x(4sin(2x)+3cos(2x))$

	$f^{(3)}(x) = -e^x(8cos(2x)-6sin(2x)) - e^x(4sin(2x)+3cos(2x)) = e^x(2sin(2x)-11cos(2x))$

	$f^{(4)}(x) = e^x(4cos(2x)+2sin(22sin(2x)) + e^x(2sin(2x)-11cos(2x)) = e^x(24sin(2x)-7cos(2x))$ \\

	Evaluating these functions at the center point, we get: \\

	$f(0) = e^0cos(0) = 1$

	$f^{(1)}(0) = e^0(cos(0)-2sin(0)) = 1$

	$f^{(2)}(0) = -e^0(4sin(0)+3cos(0)) = -3$

	$f^{(3)}(0) = e^0(2sin(0)-11cos(0)) = -11$

	$f^{(4)}(0) = -e^0(24sin(0)-7cos(0) = -7$ \\

	We now have all the pieces we need to determine the polynomial expansion. \\

	$$P_n(x) = \frac{x^0}{0!} f(0) + \frac{x^1}{1!} f^{(1)}(0) + \frac{x^2}{2!} f^{(2)}(0) +
	\frac{x^3}{3!} f^{(3)}(0) + \frac{x^4}{4!} f^{(4)}(0)$$

	$$= 1+x-\frac{3}{2} x^2 - \frac{11}{6} x^3 - \frac{7}{24} x^4$$ \\


\item

	By definition, the derivitive form of the remainder is

	$$R_n(x) = \frac{(x-x_0)^{n+1}}{(n+1)!} f^{(n+1)}(c)$$

	where $c$ is close to $x_0$.

	In this case, we have $n=4, x_0 = 0$ from the previous question.

	\newpage

	$R_n(x) = \frac{x^5}{5!}f^{(5)}(c)$

	$f^{(5)}(c) = e^x(48cos(2x) + 14sin(2x)) + e^x(24sin(2x) - 7cos(2x))$

	$= e^x(41cos(2x)+38sin(2x))$]$_{x=c}$ \\

	Therefore,

	$$R_4(x) = \frac{x^5}{120} e^c(41cos(2c)+38sin(2c))$$ \\

\item

	We know the remainder, $R_4(x)$ from part (c). Given a closed interval, we can find the maximum error by
	finding the maximum remainder in this interval. To do this, we must find its derivative,
	whose zeroes will show all maxima and minima.

	$f^{(6)}(x) = e^x(-82sin(2x)+76cos(2x)) + e^x(41cos(2x) + 38sin(2x)) = e^x(117cos(2x) - 44sin(2x))$ \\

	Now find zeroes.

	$e^x(117cos(2x)-44sin(2x)) = 0$

	$117cos(2x)-44sin(2x) = 0$

	$\frac{117}{44} = \frac{sin(2x)}{cos(2x)}$

	$x = \frac{1}{2} tan^{-1}(\frac{117}{44}) \approx 0.606$ \\

	Our candidate points are then $\{-\frac{\pi}{4}, 0.606, \frac{\pi}{4} \}$ \\

	$f^{(5)}(\frac{\pi}{4}) = 38*e^{\frac{\pi}{4}} \approx 83.3$

	$f^{(5)}(-\frac{\pi}{4}) = -38*e^{\frac{\pi}{4}}$

	$f^{(5)}(0.606) = e^{0.606}(41cos(2*0.606)+38sin(2*0.606)) \approx 91.6$ \\

	As $f^{(5)}(0.606)$ is clearly the max, this is what we use for determining the error bound. \\

	$$max(R_4(x)) = \frac{(\frac{\pi}{4})^5}{5!}*91.6$$

	$$\approx 0.228$$

\newpage

\item

Figure \ref{fig:graph1} shows the actual function compared to its $4^{th}$ degree Taylor expansion over the range
$x \in [\frac{-\pi}{4} , \frac{\pi}{4} ]$. Figure \ref{fig:graph2} shows the maximum error between the two, which
agrees with the upper bound calculated in part (c).

\begin{figure}[h!]
	\centering
	\includegraphics[width=4in, height=3.5in]{tf.png}
	\caption{Taylor expansion and initial function}
	\label{fig:graph1}
\end{figure}

\begin{figure}[!h]
	\centering
	\includegraphics[width=4in, height=3.5in]{abs_error.PNG}
	\caption{Difference between the Taylor expansion and the initial function}
	\label{fig:graph2}
\end{figure}

\eenum

% Problem 2
\item

	$b = 10.\overline{110}_2$
	$= 2^1 + (0.\overline{110})_2$

	$z = 0.\overline{110}_2$

	$z = 0.110\overline{110})2$

	$2^3*z = 110.\overline{110}_2 = 2^2 + 2^1 + z$

	$8z = 6+z$

	$z = \frac{6}{7}$

	$b_{10} = 2 + z_{10} = 2 + \frac{6}{7} = \frac{18}{7} = 2.\overline{857142}$

% Problem 3
\item


	$6.7_{10}$

	$6_{10} = 110_2$

	\begin{tabular}{ll}

		$\frac{14}{10} = d_1 + \frac{d_2}{2} + \cdots$ & $ d_1 = 1$ \\

		$\frac{4}{5} = d_2 + \frac{d_3}{2} + \cdots$ & $d_2 = 0$ \\

		$\frac{8}{5} = d_3 + \frac{d_4}{2} + \cdots$ & $d_3 = 1$ \\

		$\frac{6}{5} = d_4 + \frac{d_5}{2} + \cdots$ & $d_4 = 1$ \\

		$\frac{2}{5} = d_5 + \frac{d_6}{2} + \cdots$ & $d_5 = 0$ \\

		$\frac{4}{5} = d_6 + \frac{d_7}{2} + \cdots$ & $d_6 = 0$ \\

	\end{tabular} \\

	The value generating $d_6$ is the same as the one generating $d_2$, and thus we are in a loop generating
	the repeating pattern 0110.

	$6.7_{10} = 110.1\overline{0110}_2 = 1.101\overline{0110}*2^2$

	The IEEE representation of this number designates one signed bit (in this case it is 0, as 6.7>0), 8 for the exponent (in this case 2),
	and 23 for the mantissa (in this case $101\overline{0110}$. The $23^{rd}$ digit of the mantissa is 0, so the number is truncated when
	stored.

	The exponent that is stored is the real exponent + a bias, which in this case is 127.

	$fl(6.7)$ is stored in the machine as

	\begin{center}
	\begin{tabular}{c | c | c}

		$0$ & $10000001$ & $10101100110011001100110$ \\

		sign & exponent + bias & mantissa \\

	\end{tabular} \\
	\end{center}

	Converting the stored number back to decimal gives us $2^{129-127} * q$

	$q = 1 + 2^{-1} + 2^{-3} + 2^{-5} + 2^{-6} + 2^{-9} + 2^{-10} + 2^{-13} + 2^{-14} + 2^{-17} + 2^{-18}+2^{-21} + 2^{-22} \approx 1.675$

	$fl(6.7) = 2^2 * q  \approx (6.7+1.90735*10^{-7})$

	The relative error $d = \frac{6.7-fl(6.7)}{6.7} = 2.84679*10^{-8}$

	$\epsilon_{mach}$ for a 32 bit float is $\frac{1}{2^{23}} \approx 1.19*10^{-7}$

	$\frac{1}{2} * \epsilon_{mach} \approx 5.96 * 10^{-8}$

	Crucially, $$\frac{1}{2} * \epsilon_{mach} \leq d$$ \\

% Problem 4
\item Holmes 1.5

\benum

	\item According to the IEEE double precision standard, $16_{10}$ is stored as

	$1*2^{5} =$

	\begin{center}
	\begin{tabular}{c | c | c}

		$0$ & $10000000100$ & $0\cdots0$ \\

		sign & exponent + bias of 1023 & 52 bit mantissa \\

	\end{tabular} \\
	\end{center}

	If we let $d_n$ represent the $n^{th}$ digit in the mantissa, then the last digit stored in the machine is $d_{52}=0$. %By using
	Beucase the first 4 digits of the mantissa are used to make up the whole number 16, there are 48 binary digits available to
	represent the decimal. If the $49^{th}$ digit is a 1, and all $[d_{50}, d_{\infty})$ are 0, then the round-to-nearest rule states that
	we chop at the $48^{th}$ digit, effectively rounding down. This would make the upper bound

	$$R = 16 + 2^{-49} = 16.0000000000000017763568394002504646778106689453125$$

	The lower bound of $x$ that is recognized as 16 is $15.xxx$, or the lowest value that rounds up to 16. The whole number 15 is
	represented in binary as

	\begin{center}
	\begin{tabular}{c | c | c}

		$0$ & $10000000010$ & $1110\cdots0$ \\

		sign & exponent + bias of 1023 & 52 bit mantissa \\

	\end{tabular} \\
	\end{center}

	Unlike the upper bound, it only takes 3 bits of the mantissa to represent the whole number part of the number. There are now
	49 binary digits available to represent the decimal. Again using the round to nearest rule, if the $50^{th}$ digit is a 1, and there
	is at least one nonzero digit in $[d_{51}, d_{\infty})$, then the number is rounded up. This would make the lower bound

	$$L > 16 - 2^{-50} \rightarrow L > 15.99999999999999911182158029987476766109466552734375$$

	$$L < x \leq R \rightarrow x = 16$$ \\

	\item The same process can be used to find the upper and lower bounds of 50.

	$50 = 110010 = 1.10010*2^5$ \\

	\begin{center}
	\begin{tabular}{c | c | c}

		$0$ & $10000000100$ & $100100\cdots0$ \\

		sign & exponent + bias of 1023 & 52 bit mantissa \\

	\end{tabular} \\
	\end{center}

	In this case, the whole number part of the number takes 5 bits of the mantissa, leaving 47 digits for the decimal. Using the same
	logic as part(a) for the upper bound, we need $d_{48}$ to be 1, and $[d_{49}, d_{\infty})$ to be 0. This would make the upper bound
		$$R = 50 + 2^{-48} = 50.000000000000003552713678800500929355621337890625$$

	Here, the lower bound would be $49.xxx$, where all decimal places in the binary form are 1,again through the same logic as part(a). \\

	$49 = 110001 = 1.10001*2^5$ \\

	\begin{center}
	\begin{tabular}{c | c | c}

		$0$ & $10000000100$ & $100010\cdots0$ \\

		sign & exponent + bias of 1023 & 52 bit mantissa \\

	\end{tabular} \\
	\end{center}

	Again, 5 bits of the mantissa are reserved for the whole number 49. The smallest number that will be rounded up is one with $d_{48}=1$
	and ``$d_{\infty}=1$'', or just any number greater than $50-2^{-48}$, since there is no last digit. The lower bound is then
	$$L > 50-2^{-48} \rightarrow L > 49.999999999999996447286321199499070644378662109375$$

	$$L < x \leq R \rightarrow x = 50$$ \\

\eenum

% Problem 5
\item

	\begin{figure}[h!]
		\centering
		\includegraphics[width=4in, height=3.5in]{holmes1_12.png}
		\caption{Graph of $\frac{1-cos(x)}{x^2} $}
		\label{fig:graph333}
	\end{figure}

	It is known that $cos(0) = 1$, so as $x \rightarrow 0^+$, the numerator of this function becomes extremely small. When the machine
	subtracts these two quantities, there is a point at which catastrophic collision begins, which introduces significant error. When the
	number of ones in the binary form of $cos(x)$ is greater than 52 (the maximum number of digits in an IEEE double precision number),
	the result will dip below the underflow limit and is seen by the machine as zero.
	$$cos(10^{-8}) \approx 0.9999999999999999_{10} \approx 0.111\cdots101_2$$
	At $x=10^{-8}$, there are 54 leading ones in the binary form of $cos(x)$, so the machine computes the numerator as zero. The
	denominator, however, can still be accurately computed as $x^2$, so the result is simply 0.
	At slightly
	larger values of x ($x=10^{-7}$), there are less than 52 ones, so the machine can still do meaningful subtraction, albeit with minimal
	significance in the result.

% Problem 6
\item
Assume single-precision IEEE arithmetic. Assume that the round-to-nearest rule is used with one modification: if there is a tie then the smaller value is picked (this rule for ties is used to make the problem easier).
\benum
\item
	In order for $x>1$ to hold, the machine must see $x$ as at least $1+\epsilon_{mach}$. For a single precision machine, $\epsilon_{mach}
	=\frac{1}{2^{23}} \approx 1.19*10^{-7}$. We must now find the lowest possible value of $x$ needed for the machine to make the
	assignment. According to the round to nearest rule, a float will be rounded up if the binary representation of $x$ has a 1 in the
	$24^{th}$ decimal place, and digits $[25, \infty)$ hold at least one nonzero digit. The smallest binary value of $x$ that satisfies
	these requirements is $$x_1 = 1.000\ 000\ 000\ 000\ 000\ 000\ 000\ 001\ 0\cdots1_2$$
	As this number contains an infinite number of zeros after between the final two ones, we can only say that $x$ must be greater
	than $x_1$. Therefore,
	$$x>1+2^{-24} \rightarrow x>1.000000059604644775390625$$\\

	In order for $x<2$ to hold, the machine must see $x$ as at most $2-\epsilon_{mach}$. The maximum possible value of $x$ that is seen
	as $2-\epsilon_{mach}$ is the upper bound on the inequality. According to the round to nearest rule, a float will be rounded down if
	the final digit in the binary representation of $x$ has a 1 in the $24^{th}$ decimal place, and digits $[25, \infty)$ are all zero.
	This is the maximum possible value that can be rounded down, which can be written as
	$$x_2 = 2-2^{-24} = 1.999999940395355224609375$$
The values for which $1<x<2$ will hold on a single precision machine are $$(1.000000059604644775390625, 1.999999940395355224609375]$$ \\

\item
	In order for $x=4$ to hold, $x$ must be within the machine's ``margin of error''. The lower bound of this number $(3.999\cdots)$ will
	take up 1 digit in the mantissa for the whole number, while the upper bound $(4.000\cdots)$ will take up 2 digits. The lower bound,
	$x_1$ will then be rounded up if it's $23^{rd}$ decimal digit is a 1, and there is one other nonzero digit after it.

	$x_1 > (4-2^{-23}) \rightarrow x_1 > 3.99999988079071044921875$. \\

	The maximum possible value of $x$ that the machine will store as 4 would be $4+2^{-22}$, as a 1 in the $22^{nd}$ decimal place
	of the binary representation, followed by all zeros is rounded down.

	$x_2 = 4+2^{-22} \rightarrow x_2 = 4.0000002384185791015625$.

	$$x_1 < x \leq x_2 \rightarrow x=4$$ \\

\item

	This number is not possible because $\sqrt{2}$ is irrational, so it containts an infinite number of nonrepeating decimal digits.
	Because machines are finite precision, no computer can fully represent $\sqrt{2}$ without any rounding or approximation. \\

	When converted to binary, $\sqrt{2}_2 = 1.011\ 010\ 100\ 000\ 100\ 111\ 100\ 11\cdots$. Therefore, $x^*_l$ is the truncated form
	of the number, and $x^*_r$ the truncated form of $\sqrt{2} + 2^{-23}$

	$$x^*_l = 1.011\ 010\ 100\ 000\ 100\ 111\ 100\ 11,\ x^*_r = 1.011\ 010\ 100\ 000\ 100\ 111\ 101\ 00$$

	\begin{center}
	\begin{tabular}{cc}
		&$1.01101010000010011110100$ \\
		$-$ & $1.01101010000010011110011$ \\
		\hline
		&$0.00000000000000000000001$ \\
	\end{tabular}
	\end{center}

	$$x^*_r - x^*_l = 2^{-23} = 1.1920928955078125×10^-7$$ \\

\eenum

% Problem 7
\item
\benum
\item A problem is ill-conditioned if its solution is highly sensitive to small changes in the input data.  True or False? \\

	True. An ill-conditioned problem will grow a slight error in input data to a large error in later stages \\

\item Using higher-precision arithmetic will make an ill-conditioned problem better conditioned.  True or False? \\

	False. While the initial input data may not be as incorrect, an ill-conditioned problem will still grow that error until the result
	is not significant. \\

\item  If two real numbers are exactly representable as floating-point numbers on a finite-precision machine, then so is their product.  True or False? \\

	False. If the two numbers are small enough, i.e. $2^{-23} * 2^{-23}$ on a single-precision machine, then their product will be
	smaller than the underflow limit of the machine, which cannot be exactly represented. If denormal numbers are included, this will
	still hold true, but the underflow limit will be much lower.
\item  Consider the sum
\[
S = \frac{1}{x+1} + \frac{1}{x-1}, \quad x \ne 1.
\]
For what range of values is it difficult to compute $S$ accurately in a finite-precision system?  How will you rearrange the terms in $S$ so that the difficulty disappears? \\

This sum would be difficult to compute for $1 < x < 1+\epsilon_{mach}$, and for $-1-\epsilon_{mach} < x < -1$. When this is true,
the machine cannot accurately carry out the subtraction, the significance is lost, and the result is zero. To avoid this, some algebraic
manipulation is needed.

$$S = \frac{(x+1)+(x-1)}{(x+1)(x-1)} = \frac{2x}{x^2-1}, \quad x \ne 1$$

Any values of $x$ that are extremely close to 1 will be increased to more significant numbers due to the squaring of $x$, thus avoiding the
catastrophic collision. \\

\item In a finite-precision system with UFL = $10^{-40},$ which of the following operations will incur an underflow?
\benum
\item $\sqrt{a^2+b^2},$ with $a=1, \; b=10^{-25}.$\\

	This operation will not incur an underflow. When $b^2$ is computed, the guard bits built into the machine will be able to represent
	the quantity $10^{-25}$. Since the result of the operation is $1.00\cdots$, there is no underflow. \\

\item $\sqrt{a^2+b^2},$ with $a=b=10^{-25}.$ \\
	This operation will incur an underflow. Because the sum $a^2 + b^2$ must be stored before the square root is carried out, the sum
	must be greater than the UFL. However, $(10^{-25})^2 + (10^{-25})^2 = 2*10^{-50}$, so the UFL is breached. \\

\item $(a\times b)/(c \times d),$ with $a=10^{-20}, \; b=10^{-25}, \; c=10^{-10}, \; d=10^{-35}.$
	This operation will incur an underflow. When the initial operations are done, $(a*b), (c*d)$, the results must be stored somewhere to
	then be used for division. Each result will be $10^{-45}$, which is below the UFL.
\eenum

\eenum

\eenum

\newpage

\bc MATLAB Code \ec

Problem 1:

	\begin{figure}[h!]
		\centering
		\includegraphics[width=6in, height=3in]{script1_1.PNG}
		\caption{Code for Question 1, part d}
		\label{fig:graph4}
	\end{figure}

	\begin{figure}[h!]
		\centering
		\includegraphics[width=5in, height=3in]{script1_2.PNG}
		\caption{Code for Question 1, part d}
		\label{fig:graph5}
	\end{figure}
\newpage
Problem 5:

	\begin{figure}[h!]
		\centering
		\includegraphics[width=4in, height=1in]{script5.PNG}
		\caption{Code for Question 5}
		\label{fig:graph6}
	\end{figure}

\end{document}
