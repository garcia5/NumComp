% \input{../def}
\input{../def}

\begin{document}

\bc Numerical Computing Summer 2017 \hfill MATH/CSCI 4800 \\ Assignment-1\ec
\bc Due in class on Tuesday May 30, 2017. \ec

\bigskip

\bigskip
\n\ul{NOTES}
\benum
\item Writing solutions in LaTeX is strongly recommended but not required.
\item Show all work.  Illegible or undecipherable solutions will be returned without grading.
\item All MATLAB scripts, output and plots should be printed out and stapled to the report.  The computation portion of the HW should include a well-commented source code listing, numerical output, and graphical output if any.  The latter should consist of properly labelled and captioned graphs.  Be sure that the code and graphs are not tacked on at the back but rather, are positioned adjacent to any analytical work on the problem under consideration.
\item The problems are worth 80 points for mathematical/numerical correctness.  Upto 20 extra points will be given for clarity of argument and quality of presentation.
\eenum

\bigskip
\bc{\bf PROBLEMS}\ec
\benum

% Problem 1
\item (Pencil-and-paper  and MATLAB)  Let $f(x) = e^{x}\cos {2x}.$
\benum
\item Find $T_4(x),$ the degree-4 Taylor polynomial of $f(x)$ centered at $x=0.$
\item Find the derivative form of the remainder $R_4(x)$ in the expression
\[
f(x) - T_4(x) = R_4(x).
\]
\item Suppose that $f(x)$ is approximated by $T_4(x)$ in the interval $x \in [-\pi/4,\pi/4].$  Find a bound on the absolute error of the approximation.
\item Using Matlab, plot $f(x)$ and $T_4(x)$ on the same graph for $x \in [-\pi/4,\pi/4].$
On a second graph plot the absolute error $|f(x)-T_4(x)|,$ again for $x \in [-\pi/4,\pi/4].$  Check if the error does indeed obey the bound you found in part (c) above.
\eenum

% Problem 2
\item (Pencil-and-paper)
Convert the binary number $10.\overline{110}$ to base $10.$  Give the answer both as a decimal and a fraction if you can.

% Problem 3
\item (Pencil-and-paper)
Express $x=6.7$ as an IEEE single-precision float $\text{fl}(x)$ using the round-to-nearest rule.  Compute the relative error $d=|x-fl(x)|/|x|$ exactly as a base-10 number, and show that $d$ satisfies $d \le (1/2)\e_{mach}.$


% Problem 4
\item (Pencil-and-paper and MATLAB)
Holmes 1.5(a).

% Problem 5
\item (Pencil-and-paper and MATLAB)
Holmes 1.12.

% Problem 6
\item
(Pencil-and-paper, adapted from Holmes Problem 1.16.)  Assume single-precision IEEE arithmetic. Assume that the round-to-nearest rule is used with one modification: if there is a tie then the smaller value is picked (this rule for ties is used to make the problem easier).
\benum
\item For what real numbers $x$ will the computer claim that the inequalities $1 < x < 2$ hold?
\item For what real numbers $x$ will the computer claim $x = 4?$
\item Suppose it is stated that there is a floating point number $x^*$ that is the exact solution of
$x^2-2=0.$   Why is this not possible? Also, suppose $x^*_L$ and $x^*_R$ are the floats to the left and right of $\sqrt 2$ respectively.  What is the value of $x^*_R-x^*_L?$
\eenum

% Problem 7
\item
\benum
\item A problem is ill-conditioned if its solution is highly sensitive to small changes in the input data.  True or False?
\item Using higher-precision arithmetic will make an ill-conditioned problem better conditioned.  True or False?
\item  If two real numbers are exactly representable as floating-point numbers on a finite-precision machine, then so is their product.  True or False?
\item  Consider the sum
\[
S = \frac{1}{x+1} + \frac{1}{x-1}, \quad x \ne 1.
\]
For what range of values is it difficult to compute $S$ accurately in a finite-precision system?  How will you rearrange the terms in $S$ so that the difficulty disappears?
\item In a finite-precision system with UFL = $10^{-40},$ which of the following operations will incur an underflow?
\benum
\item $\sqrt{a^2+b^2},$ with $a=1, \; b=10^{-25}.$
\item $\sqrt{a^2+b^2},$ with $a=b=10^{-25}.$
\item $(a\times b)/(c \times d),$ with $a=10^{-20}, \; b=10^{-25}, \; c=10^{-10}, \; d=10^{-35}.$
\eenum

\eenum

\eenum
\end{document}
