\documentclass[11pt]{article}
\usepackage{color, array, graphics}
\usepackage{amsmath}
\usepackage{amssymb}
\usepackage{enumerate}
\usepackage{mathtools}
\usepackage{fullpage}
\usepackage{graphicx}
\usepackage{float}
\usepackage{listings}
\usepackage[utf8]{inputenc}

\begin{document}
\lstset{stringstyle=\ttfamily,
	showstringspaces=false,
	basicstyle=\small}

\begin{center} Alexander Garcia \hfill June 12, 2017 \\ Assignment-3 \end{center}

\medskip

\begin{enumerate}

	\item
	\begin{enumerate}[(a)]

		\item \textbf{False}.

		Because $\mathbf{A}$ is nonsingular, there are a number of properties that are associated with it. One of these is the fact that $\mathbf{A}$ being
		singular implies that $\mathbf{Ax = b}$ has a unique solution for all $\mathbf{b}$. Likewise if $\mathbf{b = 0}$, then $\mathbf{x}$ is necessarily the 0
		vector.

		This question was done assuming that $\mathbf{A}=n\times n \rightarrow \mathbf{b}=1\times n$. If $\mathbf{b}$ is of size $m<n$, then $\mathbf{Ax=b}$ has
		an infinity of solutions. However, this does not depend on the contents of $\mathbf{b}$, but rather the initial parameters of the problem.

		\item \textbf{False}.

		This can be proven false simply by taking the determinant of a $3\times 3$ matrix.
		$$det(\mathbf{A}) = a_{11}(a_{22}a_{33} + a_{23}a_{32}) - a_{12}(a_{21}a_{33} + a_{23}a_{31}) + a_{13}(a_{21}a_{32} + a_{22}a_{31})$$

		As there are terms in this equation that are not dependant on $a_{ii}$, a matrix with a zero on the principle diagonal is not necessarily singular.\\

		\item \textbf{False}.

		The conditioning of a matrix $\mathbf{A}$ is related to the largest sum of a single column (the 1-norm). The partial pivoting algorithm scans the rest of the
		column below the current pivot point $(a_{kk})$ for the largest value.

		A well-conditioned matrix would simply have small column sums, which does not necessarily affect the location of the largest values in each row, which in turn
		does not affect the need for partial pivoting. \\

		\item \textbf{True}.

		By definition, the condition number of a matrix $\mathbf{A = ||A||\ ||A^{-1}||}$. Therefore, the condition number of
		$\mathbf{A^{-1} = ||A^{-1}||\ ||(A^{-1})^{-1}|| = ||A^{-1}||\ ||A||}$. As $||A||,\ ||A^{-1}||$ is a scalar value, the two products are the same.

		\item \textbf{False}.

		When working with a linear system, the solution to the system is the point at which every line defined by matrix $\mathbf{A}$ intersects. If there is no
		point where every line intersects, then there are no solutions. Because the system is
		linear, two lines can have only one discreet intersection. Therefore, it is not possible for 2 (or more) lines to intersect at more than one point, so there
		must be either exactly 1 solution, or infinitely many solutions. \\

	\item

	\end{enumerate}

\end{enumerate}

\end{document}
